\chapter{Einfuehrung}

\section{Systemprogrammierung}

\marginnote{Definiere Systemprogrammierung!}
Konstruktion von Algorithmen fuer ein Rechensystem, die die Bearbeitung und Ausfuehrung von Benutzerprogrammen unter Qualitaetskriterien organisieren (steuern, kontrollieren) und z.T. selbst ausfuehren.

\marginnote{Nenne die Qualitaetskritierien fuer die Systemprogrammierung!}
\begin{enumerate}
	\item Zuverlaessigkeit
	\item Effizienz und Performance
	\item Realzeitanforderungen
	\item Sicherheitsanforderungen
	\item Benutzerfreundlichkeit
\end{enumerate}

\marginnote{Welche Betriebsarten gibt es fuer BS?}
\begin{enumerate}
	\item Stapelverarbeitung (batch processing)
	\item Dialogbetrieb (time sharing)
	\item Transaktionsbetrieb (transaction system)
	\item Echtzeitbetrieb (real time system)
\end{enumerate}

\marginnote{Welche Faktoren beeinflussen die Entwicklung von BS?}
\begin{enumerate}
	\item Fortschritte der Hrdwaretechnologie (Preis-Leistungs Verhaeltnis)
	\item Dialogbetrieb (time sharing)
	\item Uebergang von numerischer Berechnung zur allgemeinen Informationsverarbeitung
	\item Neue Anwendungsbereiche und Oeffnung fuer Nichtspezialisten 
\end{enumerate}