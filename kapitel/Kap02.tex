\chapter{Parallele Systeme - Modellierung, Strukturen}

\section{Motivation}

\marginnote{Was sind die Grenzen des Moore's Law? Was ist die Loesung?}
\begin{itemize}
	\setlength\itemsep{0em}
	\item Moore's law: Anzahl der Transistoren verdoppelt sich alle 18 Monate
	\item Physikalische Grenze - Power Wall: Hoehere CPU Frequenz bedeutet zumende Verlustleistung durch mangelnde Kuehlleistung, Hardware kommt an die Grenzen
	\item Memory Wall: Speicherzugriff ist bottle neck, CPU-Frequenz $>$ Speicherfrequenz
	\item Instruction Level Parallelism Wall (ILP): Design-Komplexitaet - Architekturen der Mikroprozessoren
	\item Loesung: Parallele und Verteilte Systeme, sodass sich die Aktivitaet nicht um eine Ressource streiten muss. 
\end{itemize}

\section{Sequentielle und Parallele Systeme}

\section{Begriffliche Grundlagen}

\section{Beschreibungskonzepte paralleler Aktivitäten}

\section{Modellierung paralleler Systeme}

\section{Ereignisse und Aktionsstrukturen}

\section{Aktionen als Zustandsübergänge}

\section{Petri-Netze}
