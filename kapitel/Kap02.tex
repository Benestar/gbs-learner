\chapter{Parallele Systeme - Modellierung, Strukturen}
\marginnote{Was sind die Grenzen des Moore's Law? Was ist die Loesung?}
\begin{itemize}
	\setlength\itemsep{0em}
	\item Moore's law: Anzahl der Transistoren verdoppelt sich alle 18 Monate
	\item Physikalische Grenze - Power Wall: Hoehere CPU Frequenz bedeutet zumende Verlustleistung durch mangelnde Kuehlleistung, Hardware kommt an die Grenzen
	\item Memory Wall: Speicherzugriff ist bottle neck, CPU-Frequenz $>$ Speicherfrequenz
	\item Instruction Level Parallelism Wall (ILP): Design-Komplexitaet - Architekturen der Mikroprozessoren
	\item Loesung: Parallele und Verteilte Systeme, sodass sich die Aktivitaet nicht um eine Ressource streiten muss. 
\end{itemize}

\section{Sequentielle und parallele Systeme}
\fatmarginnote{Was ist grunds. ein Programm?}
Ein Programm ist eine Repräsentation eines Algorithmus und ist eine sequentielle Folge von Anweisungen. \\
\fatmarginnote{Was erhält man bei seq. Programmen?}
\fatcapitals{Determinierte Programme} Programm produzierts stets gleiche Ergebnisse bei gleichen Bedingungen und Eingaben. \fatcapitals{Determinierte Abläufe} Keine willkürliche / zufällige Auswahl von Schritten, sondern \fatcapitals{eindeutig vorbestimmter Ablauf}

\section{Petri-Netze}
\fatmarginnote{Erreichbarkeit}


\fatmarginnote{Lebendigkeit}

\fatmarginnote{Verklemmung}