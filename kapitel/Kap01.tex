\chapter{Einfuehrung}
\section{Grundlegende Begriffe}


\section{Betriebssysteme: Hauptaufgaben}
\marginnote{Anlass?}
\noindent Schutz des BS und der Hardware vor Fehlern und Angriffen

\marginnote{Welche Loesungsansaetze gibt es? Welche Klassifizierung in Arbeitsmodi resultiert?}
\begin{longtable}{ll}
	Benutzermodus (User Mode / Space)& Systemmodus (Kernel Mode / Space)\\
	Kein direkter Hardware Zugriff moeglich& Privilegierter Modus\\ 
\end{longtable}


\section{Systemprogrammierung}

\marginnote{Definiere Systemprogrammierung!}
\noindent Konstruktion von Algorithmen fuer ein Rechensystem, die die Bearbeitung und Ausfuehrung von Benutzerprogrammen unter Qualitaetskriterien organisieren (steuern, kontrollieren) und z.T. selbst ausfuehren.

\marginnote{Nenne die Qualitaetskritierien fuer die Systemprogrammierung!}
\begin{enumerate}
	\setlength\itemsep{0em}
	\item Zuverlaessigkeit
	\item Effizienz und Performance
	\item Realzeitanforderungen
	\item Sicherheitsanforderungen
	\item Benutzerfreundlichkeit
\end{enumerate}

\marginnote{Welche Betriebsarten gibt es fuer BS?}
\begin{enumerate}
	\setlength\itemsep{0em}
	\item Stapelverarbeitung (batch processing)
	\item Dialogbetrieb (time sharing)
	\item Transaktionsbetrieb (transaction system)
	\item Echtzeitbetrieb (real time system)
\end{enumerate}

\marginnote{Welche Faktoren beeinflussen die Entwicklung von BS?}
\begin{enumerate}
	\setlength\itemsep{0em}
	\item Fortschritte der Hrdwaretechnologie (Preis-Leistungs Verhaeltnis)
	\item Dialogbetrieb (time sharing)
	\item Uebergang von numerischer Berechnung zur allgemeinen Informationsverarbeitung
	\item Neue Anwendungsbereiche und Oeffnung fuer Nichtspezialisten 
\end{enumerate}

\section{Betriebssystem-Architekturen}

\section{Hardwarenahe Programme}
\subsection{Linker und Loader}
\marginnote{Differenziere die Begriffe Linker und Loader}
\begin{tabularx}{\textwidth}{X X}
	Linker&			Loader \\
\end{tabularx}

\marginnote{Nenne Vor- und Nachteile eines statischen Linkers}
\begin{tabularx}{\textwidth}{X X}
	Vorteile& Nachteile\\
	alle benoetigten Routinen sind fuer die Ausfuehrung in einem Programm vorhanden, schnell, direkt ausfuehrbar& inflexibel, Speicherverschwendung, shared libraries werden individuell kopiert\\ 
\end{tabularx}
